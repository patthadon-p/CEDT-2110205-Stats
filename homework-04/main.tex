\documentclass[a4paper, 10pt]{article}
\usepackage{../CEDT-Homework-style}

\usepackage{booktabs}
\usepackage{amsmath}
\allowdisplaybreaks

\setlength{\headheight}{14.49998pt}

\begin{document}
\subject[2110205 - Statistics for Computer Engineering]
\hwtitle{4}{Week 4}{6733172621 Patthadon Phengpinij}{ChatGPT (for LaTeX styling and grammar checking)}


% ================================================================================ %
\section{Week 4: Expectation and \text{Var}iance of Discrete RVs}
% ================================================================================ %



% ================================================================================ %
%                                    Problem 01                                    %
% ================================================================================ %
\begin{problem}
A fair 10-sided die is rolled. Let \( X \) be the random variable representing the outcome of the roll.
\begin{subproblems}
    \item What is the expectation \( E[X] \)?
    \item What is the variance \( \text{Var}(X) \)?
    \item Let \( Y = 5X - 3 \). Calculate \( E[Y] \) and \( \text{Var}(Y) \).
\end{subproblems}
\end{problem}

\begin{solution}
In this situation, we have a discrete uniform distribution--\( X \sim \text{Uniform}(10) \).
For uniform distribution \( Z \sim \text{Uniform}(n) \):
\[
    E[Z] = \frac{n + 1}{2}, \quad \text{Var}(Z) = \frac{n^2 - 1}{12}
\]
Therefore,
\[
    E[X] = \frac{10 + 1}{2} = \frac{11}{2} = \boxed{5.5} \; \textbf{a).}
\]
and,
\[
    \text{Var}[X] = \frac{10^2 - 1}{12} = \frac{99}{12} = \frac{33}{4} = \boxed{8.25} \; \textbf{b).}
\]
while \( \text{range}(X) = \{1, 2, 3, \dots, 10\} \), and \( P(X = x) = \frac{1}{10} \).

\vspace{2mm}

Because,
\[
    E(aX + b) = aE(X) + b
\]
and,
\[
    \text{Var}(aX + b) = a^2 \text{Var}(X)
\]
while, \( Y = 5X - 3 \). Thus,
\begin{align*}
    E(Y) &= E(5X - 3) \\
    &= 5E(X) - 3 \\
    &= 5 \times 5.5 - 3 \\
    &= 27.5 - 3 \\
    &= 24.5
\end{align*}
and,
\begin{align*}
    \text{Var}(Y) &= \text{Var}(5X - 3) \\
    &= 5^2 \cdot \text{Var}(X) \\
    &= 25 \times 8.25 \\
    &= 206.25
\end{align*}

Which means
\[\boxed{
    E(Y) = 24.5, \; \text{Var}(Y) = 206.25
} \; \textbf{c).} \]
\end{solution}
% ================================================================================ %


% ================================================================================ %
%                                    Problem 02                                    %
% ================================================================================ %
\begin{problem}
A student takes a multiple-choice quiz. For a single question, the probability of answering correctly is 0.8.
Let \( X = 1 \) if the answer is correct and \( X = 0 \) otherwise.
\begin{subproblems}
    \item Find the expectation \( E[X] \).
    \item Find the variance \( \text{Var}(X) \).
    \item Calculate \( E[X^2] \).
\end{subproblems}
\end{problem}

\begin{solution}
This distribution is a Bernoulli distribution--\( X \sim \text{Bernoulli}(0.8) \).
The expected value and variance of \( Y \sim \text{Bernoulli}(p) \) are:
\[
    E[Y] = p, \quad \text{Var}(Y) = p(1 - p)
\]
Thus,
\[
    E[X] = \boxed{0.8} \; \textbf{a).}
\]
and,
\[
    \text{Var}(X) = 0.8(1 - 0.8) = 0.8 \cdot 0.2 = \boxed{0.16} \; \textbf{b).}
\]
Because, \( \text{Var}(X) = E[X^2] - E[X]^2 \), so that:
\begin{align*}  
    E[X^2] &= \text{Var}(X) + E[X]^2 \\
    &= 0.16 + (0.8)^2 \\
    &= 0.16 + 0.64 \\
    &= \boxed{0.8} \; \textbf{c).}
\end{align*}
\end{solution}
% ================================================================================ %


% ================================================================================ %
%                                    Problem 03                                    %
% ================================================================================ %
\begin{tosubmit}
\problem
A manufacturer produces computer chips, and 5\% of them are defective.
A sample of 20 chips is selected for testing.
Let \( Y \) be the number of defective chips in the sample.
\begin{subproblems}
    \item What is the probability of finding exactly 2 defective chips?
    \item What is the expected number of defective chips, \( E[Y] \)?
    \item What is the variance of the number of defective chips, \( \text{Var}(Y) \)?
\end{subproblems}

\par\noindent\submitsolution According to the statement, \( Y \sim \text{Binomial}(20, 0.05) \)

\vspace{2mm}

\par\noindent\textbf{a).} \, The probability of finding exactly 2 defective chips
can be calculated using the binomial probability formula:
\[
    P(Y = 2) = \binom{20}{2} (0.05)^{2} (0.95)^{18} \approx \boxed{0.1887}
\]

\vspace{2mm}

\par\noindent\textbf{b).} \, The expected number of defective chips, \( E[Y] \),
is the expected value of a binomial distribution, \( Y \sim \text{Binomial}(20, 0.05) \):
\[
    E(Y) = (20)(0.05) = \boxed{1}
\]

\vspace{2mm}

\par\noindent\textbf{c).} \, The variance of  number of defective chips, \( \text{Var}[Y] \),
is the variance of a binomial distribution, \( Y \sim \text{Binomial}(20, 0.05) \):
\[
    E(Y) = (20)(0.05)(1-0.05) = \boxed{0.95}
\]
\end{tosubmit}
% ================================================================================ %

\pagebreak

% ================================================================================ %
%                                    Problem 04                                    %
% ================================================================================ %
\begin{problem}
The average number of calls arriving at a customer service center is 8 calls per hour.
Let \( X \) be the number of calls in a given hour.
\begin{subproblems}
    \item Find the expectation \( E[X] \).
    \item Find the variance \( \text{Var}(X) \).
    \item Calculate \( E[X^2] \).
\end{subproblems}
\end{problem}

\begin{solution}
This distribution is a Poisson distribution--\( X \sim \text{Poisson}(8) \).
The expected value and variance of \( Y \sim \text{Poisson}(\lambda) \) are:
\[
    E[Y] = \lambda, \quad \text{Var}(Y) = \lambda
\]
Thus,
\[
    E[X] = \boxed{8} \; \textbf{a).}
\]
and,
\[
    \text{Var}(X) = \boxed{8} \; \textbf{b).}
\]
From the relation, \( \text{Var}(X) = E[X^2] - E[X]^2 \), we can compute:
\begin{align*}
    E[X^2] &= \text{Var}(X) + E[X]^2 \\
    &= 8 + (8)^2 \\
    &= 8 + 64 \\
    &= \boxed{72} \; \textbf{c).}
\end{align*}
\end{solution}
% ================================================================================ %


% ================================================================================ %
%                                    Problem 05                                    %
% ================================================================================ %
\begin{problem}
Let X be a random variable with \( E[X] = \) 6 and \( \text{Var}(X) = \) 2.
Let \( Y = -2X + 7 \).
\begin{subproblems}
    \item Find \( E[Y] \).
    \item Find \( \text{Var}(Y) \).
    \item Calculate \( E[X^2] \).
\end{subproblems}
\end{problem}

\begin{solution}
Because \( Y = -2X + 7 \):
\begin{align*}
    E[Y] &= -2E[X] + 7 \\
    &= -2(6) + 7 \\
    &= -12 + 7 \\
    &= \boxed{-5} \; \textbf{a).}
\end{align*}
and,
\begin{align*}
    \text{Var}(Y) &= (-2)^2 \text{Var}(X) \\
    &= 4(2) \\
    &= \boxed{8} \; \textbf{b).}
\end{align*}
From the relation, \( \text{Var}(Y) = E[Y^2] - E[Y]^2 \), we can compute \( E[Y^2] \):
\begin{align*}
    E[Y^2] &= \text{Var}(Y) + E[Y]^2 \\
    &= 8 + (-5)^2 \\
    &= 8 + 25 \\
    &= \boxed{33} \; \textbf{c).}
\end{align*}
\end{solution}
% ================================================================================ %


% ================================================================================ %
%                                    Problem 06                                    %
% ================================================================================ %
\begin{tosubmit}
\problem[6]
A number \( X \) is chosen at random from the set \( \{1, 2, 3, \dots , 12\} \).
A prize is awarded based on the formula \( P = 3X^2 - 5 \).
\begin{subproblems}
    \item Find the expected value of the prize, \( E[P] \).
    \item Find the variance of \( X \).
\end{subproblems}
\( \paren{\text{Hint: } \sum_{i=1}^n i = \frac{n(n+1)}{2} \text{ and } \sum_{i=1}^n i^2 = \frac{n(n+1)(2n+1)}{6}} \)

\vspace{2mm}

\par\noindent\submitsolution
\[
    E[P] = E[3X^2 - 5] = 3E[X^2] - 5
\]

\vspace{2mm}

\par\noindent While \( X \sim \text{Uniform}(12) \), the expectation of Uniform Distribution is:
\[
    E[X] = \frac{n+1}{2} = \frac{12+1}{2} = 6.5
\]
and the variance of Uniform Distribution is:
\[
    \text{Var}(X) = \frac{n^2 - 1}{12} = \frac{12^2 - 1}{12} = \boxed{\frac{143}{12}} \; \text{b).}
\]
We can calculate \( E[X^2] \) as follows:
\begin{align*}    
    \text{Var}(X) &= E[X^2] - \paren{E[x]}^2 \\
    E[X^2] &= \text{Var}(X) + \paren{E[x]}^2 \\
    &= \frac{143}{12} + (6.5)^2 \\
    &= \frac{143}{12} + \frac{169}{4} \\
    &= \frac{650}{12} \\
    E[X^2] &= \frac{325}{6}
\end{align*}
we know that:
\begin{align*}
  E[P] &= 3E[X^2] - 5 \\
  &= 3 \times \frac{325}{6} - 5 \\
  &= \frac{325}{2} - 5 \\
  &= \frac{315}{2} \\
  &= \boxed{157.5} \; \text{a).}
\end{align*}
\end{tosubmit}
% ================================================================================ %

\pagebreak

% ================================================================================ %
%                                    Problem 07                                    %
% ================================================================================ %
\begin{problem}
A scientist runs an experiment that succeeds with probability \( p = 0.4 \).
The experiments are run until one is successful. The cost of each experiment is \$100.
Let \( N \) be the number of experiments run.
\begin{subproblems}
    \item What is the expected total cost, \( E[100N] \)?
    \item What is the variance of the total cost, \( \text{Var}(100N) \)?
\end{subproblems}
\end{problem}

\begin{solution}
The random variable \( N \) follows a geometric distribution with parameter \( p = 0.4 \).

The expected value of a geometrically distributed random variable is given by:
\[
    E[N] = \frac{1}{p} = \frac{1}{0.4} = 2.5
\]
Thus, the expected total cost is:
\[
    E[100N] = 100E[N] = 100 \times 2.5 = \boxed{250} \; \textbf{a).}
\]

The variance of a geometrically distributed random variable is given by:
\[
    \text{Var}(N) = \frac{1-p}{p^2} = \frac{0.6}{(0.4)^2} = \frac{0.6}{0.16} = 3.75
\]
Thus, the variance of the total cost is:
\[
    \text{Var}(100N) = 100^2 \text{Var}(N) = 10000 \times 3.75 = \boxed{37500} \; \textbf{b).}
\]
\end{solution}
% ================================================================================ %


% ================================================================================ %
%                                    Problem 08                                    %
% ================================================================================ %
\begin{problem}
A factory has two machines. Machine 1 produces items with defects at a Poisson rate of 0.8 defects per item.
Machine 2 produces items with defects at a Poisson rate of 1.2 defects per item.
Machine 1 produces 60\% of the factory's output, and Machine 2 produces 40\%.
An item is selected at random and found to have exactly 1 defect.
\begin{subproblems}
    \item What is the probability that this item was produced by Machine 1?
\end{subproblems}
\end{problem}

\begin{solution}
Let \( D_1 \sim Poisson(0.8) \) be the event that the item was produced by Machine 1,
and \( D_2 \sim Poisson(1.2) \) be the event that it was produced by Machine 2.

\vspace{2mm}

\par\noindent We want to find \( P(D_1 | 1 \text{ defect}) \).

Using Bayes' theorem:
\[
P(D_1 | 1 \text{ defect}) = \frac{P(1 \text{ defect} | D_1) P(D_1)}{P(1 \text{ defect})}
\]

We know:
\begin{align*}
    P(D_1) &= 0.6 \\
    P(D_2) &= 0.4
\end{align*}

Next, we need to find \( P(1 \text{ defect} | D_1) \) and \( P(1 \text{ defect} | D_2) \).

For Machine 1:
\[
P(1 \text{ defect} | D_1) = \frac{(0.8)^1 e^{-0.8}}{1!} = 0.8 e^{-0.8}
\]

For Machine 2:
\[
P(1 \text{ defect} | D_2) = \frac{(1.2)^1 e^{-1.2}}{1!} = 1.2 e^{-1.2}
\]

Now we can find \( P(1 \text{ defect}) \):
\begin{align*}
    P(1 \text{ defect}) &= P(1 \text{ defect} | D_1) P(D_1) + P(1 \text{ defect} | D_2) P(D_2) \\
    &= (0.8 e^{-0.8})(0.6) + (1.2 e^{-1.2})(0.4)
\end{align*}

Substituting these values back into Bayes' theorem:
\[
P(D_1 | 1 \text{ defect}) = \frac{(0.8 e^{-0.8})(0.6)}{(0.8 e^{-0.8})(0.6) + (1.2 e^{-1.2})(0.4)}
\]

This simplifies to:
\[
P(D_1 | 1 \text{ defect}) = \frac{0.48 e^{-0.8}}{0.48 e^{-0.8} + 0.48 e^{-1.2}} = \frac{e^{-0.8}}{e^{-0.8} + e^{-1.2}}
\approx \boxed{0.599} \; \textbf{a).}
\]
\end{solution}
% ================================================================================ %


% ================================================================================ %
%                                    Problem 09                                    %
% ================================================================================ %
\begin{problem}
There are two bags of marbles. Bag A contains 3 red and 7 blue marbles.
Bag B contains 6 red and 4 blue marbles. You choose a bag at random and draw one marble. The marble is red.
\begin{subproblems}
    \item What is the probability that the marble came from Bag A?
\end{subproblems}
\end{problem}

\begin{solution}
Let \( A \) be the event that the marble was drawn from Bag A,
and \( B \) be the event that the marble was drawn from Bag B.

\vspace{2mm}

We want to find \( P(A | \text{red}) \).

Using Bayes' theorem:
\[
P(A | \text{red}) = \frac{P(\text{red} | A) P(A)}{P(\text{red})}
\]

We know:
\begin{align*}
    P(A) &= 0.5 \\
    P(B) &= 0.5
\end{align*}

Next, we need to find \( P(\text{red} | A) \) and \( P(\text{red} | B) \).

For Bag A:
\[
P(\text{red} | A) = \frac{3}{10}
\]

For Bag B:
\[
P(\text{red} | B) = \frac{6}{10}
\]

Now we can find \( P(\text{red}) \):
\begin{align*}
    P(\text{red}) &= P(\text{red} | A) P(A) + P(\text{red} | B) P(B) \\
    &= \left(\frac{3}{10}\right)\left(0.5\right) + \left(\frac{6}{10}\right)\left(0.5\right) \\
    &= \frac{3}{20} + \frac{6}{20} = \frac{9}{20}
\end{align*}

Substituting these values back into Bayes' theorem:
\[
P(A | \text{red}) = \frac{\left(\frac{3}{10}\right)\left(0.5\right)}{\frac{9}{20}} = \frac{\frac{3}{20}}{\frac{9}{20}} = \frac{3}{9} = \frac{1}{3}
\]

Thus, the probability that the marble came from Bag A is \( \boxed{\frac{1}{3}} \).
\end{solution}
% ================================================================================ %


% ================================================================================ %
%                                    Problem 10                                    %
% ================================================================================ %
\begin{problem}
You roll a fair six-sided die. Let the outcome be \( K \).
You then flip a biased coin \( K \) times. The coin has a probability of heads of 0.7.
Let \( H \) be the number of heads obtained.
\begin{subproblems}
    \item Find the expected number of heads, \( E[H] \).
    \item Find the variance of the number of heads, \( \text{Var}(H) \).
\end{subproblems}
\end{problem}

\begin{solution}
We can use the information given by the statement to conclude that:
\[
    K \sim \text{Uniform}(6), \; H | K \sim \text{Binomial}(K, 0.7)
\]
\par\noindent\textbf{a).} From the law of total expectation:
\[
    E[H] = E[E[H | K]]
\]
Thus,
\begin{align*}
    E[H] &= E[E[H | K]] \\
    &= E[K \times 0.7] \\
    &= 0.7 \times E[K] \\
    &= 0.7 \times \frac{6 + 1}{2} \\
    &= 0.7 \times \frac{7}{2} \\
    &= 0.7 \times 3.5 \\
    &= \boxed{2.45}
\end{align*}

\par\noindent\textbf{b).} From the law of total variance:
\[
    \text{Var}(H) = E[\text{Var}(H | K)] + \text{Var}(E[H | K])
\]
Thus,
\begin{align*}
    \text{Var}(H) &= E[\text{Var}(H | K)] + \text{Var}(E[H | K]) \\
    &= E[K \times 0.7 \times (1 - 0.7)] + \text{Var}(K \times 0.7) \\
    &= E[K \times 0.21] + \paren{(0.7)^2 \times \text{Var}(K)} \\
    &= \paren{0.21 \times E[K]} + \paren{0.49 \times \text{Var}(K)} \\
    &= \paren{0.21 \times \frac{6 + 1}{2}} + \paren{0.49 \times \frac{6^2 - 1}{12}} \\
    &= \paren{0.21 \times \frac{7}{2}} + \paren{0.49 \times \frac{35}{12}} \\
    &= 0.735 + \paren{0.49 \times 2.9167} \\
    &= 0.735 + 1.429 \\
    &= \boxed{2.164}
\end{align*}
\end{solution}
% ================================================================================ %


\end{document}