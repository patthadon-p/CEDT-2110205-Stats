\documentclass[a4paper, 10pt]{article}
\usepackage{../CEDT-Homework-style}

\usepackage{booktabs}
\usepackage{amsmath}
\allowdisplaybreaks

\setlength{\headheight}{14.49998pt}

\begin{document}
\subject[2110205 - Statistics for Computer Engineering]
\hwtitle{4}{Week 4}{6733172621 Patthadon Phengpinij}{ChatGPT (for LaTeX styling and grammar checking)}


% ================================================================================ %
\section{Week 4: Expectation and Variance of Discrete RVs}
% ================================================================================ %



% ================================================================================ %
%                                    Problem 0x                                    %
% ================================================================================ %
\begin{problem}
A fair 10-sided die is rolled. Let \( X \) be the random variable representing the outcome of the roll.
\begin{subproblems}
    \item What is the expectation \( E[X] \)?
    \item What is the variance \( Var(X) \)?
    \item Let \( Y = 5X - 3 \). Calculate \( E[Y] \) and \( Var(Y) \).
\end{subproblems}
\end{problem}

\begin{solution}
\par\noindent\textbf{a).} \, The expectation \( E[X] \) of a fair 10-sided die can be calculated
using the formula for the expectation of a discrete random variable:
\[
    E(X) = \sum_{x \in \text{range}(X)} x P(X = x)
\]
while \( \text{range}(X) = \{1, 2, 3, \dots, 10\} \), and \( P(X = x) = \frac{1}{10} \).
Thus,
\begin{align*}
    E(X) &= \sum_{x \in \text{range}(X)} x P(X = x) \\
    &= \sum_{x \in \{1, 2, 3, \dots, 10\}} x \cdot \frac{1}{10} \\
    &= \frac{1}{10} \sum_{x=1}^{10} x \\
    &= \frac{1}{10} \cdot \frac{10(10 + 1)}{2} \\
    &= \frac{11}{2} \\
    E(X) &= \boxed{5.5}
\end{align*}

\vspace{2mm}

\par\noindent\textbf{b).} \, The variance \( Var(X) \) of a fair 10-sided die can be calculated
using the formula for the variance of a discrete random variable:
\[
    Var(X) = E\sqbracket{\paren{X - E[X]}^2} = E(X^2) - (E(X))^2
\]
while \( E(X) = 5.5 \), we need to calculate \( E(X^2) \):
\begin{align*}
    E(x^2) &= \sum_{x=1}^{10} x^2 P(X = x) \\
    &= \frac{(10)(10 + 1)(2 \times 10 + 1)}{6} \times \frac{1}{10} \\
    &= \frac{(10)(11)(21)}{6} \times \frac{1}{10} \\
    &= 38.5
\end{align*}
So that, \( Var(X) = E(X^2) - (E(X))^2 = 38.5 - \paren(5.5)^2 = 38.5 - 30.25 = \boxed{8.25} \).

\vspace{2mm}

\par\noindent\textbf{c).} \, Becase,
\[
    E(aX + b) = aE(X) + b
\]
and,
\[
    Var(aX + b) = a^2 Var(X)
\]
while, \( Y = 5X - 3 \). Thus,
\begin{align*}
    E(Y) &= E(5X - 3) \\
    &= 5E(X) - 3 \\
    &= 5 \times 5.5 - 3 \\
    &= 27.5 - 3 \\
    &= 24.5
\end{align*}
and,
\begin{align*}
    Var(Y) &= Var(5X - 3) \\
    &= 5^2 \cdot Var(X) \\
    &= 25 \times 8.25 \\
    &= 206.25
\end{align*}

Which means
\[\boxed{
    E(Y) = 24.5, \; Var(Y) = 206.25
}\]
\end{solution}
% ================================================================================ %


% ================================================================================ %
%                                    Problem 03                                    %
% ================================================================================ %
\begin{tosubmit}
\problem[3]
A manufacturer produces computer chips, and 5\% of them are defective.
A sample of 20 chips is selected for testing.
Let \( Y \) be the number of defective chips in the sample.
\begin{subproblems}
    \item What is the probability of finding exactly 2 defective chips?
    \item What is the expected number of defective chips, \( E[Y] \)?
    \item What is the variance of the number of defective chips, \( Var(Y) \)?
\end{subproblems}

\par\noindent\submitsolution According to the statement, \( Y \sim Binomial(20, 0.05) \)

\vspace{2mm}

\par\noindent\textbf{a).} \, The probability of finding exactly 2 defective chips
can be calculated using the binomial probability formula:
\[
    P(Y = 2) = \binom{20}{2} (0.05)^{2} (0.95)^{18} \approx \boxed{0.3774}
\]

\vspace{2mm}

\par\noindent\textbf{b).} \, The expected number of defective chips, \( E[Y] \),
is the expected value of a binomial distribution, \( Y \sim Binomial(20, 0.05) \):
\[
    E(Y) = (20)(0.05) = \boxed{1}
\]

\vspace{2mm}

\par\noindent\textbf{c).} \, The variance of  number of defective chips, \( Var[Y] \),
is the variance of a binomial distribution, \( Y \sim Binomial(20, 0.05) \):
\[
    E(Y) = (20)(0.05)(1-0.05) = \boxed{0.95}
\]
\end{tosubmit}
% ================================================================================ %


% ================================================================================ %
%                                    Problem 06                                    %
% ================================================================================ %
\begin{tosubmit}
\problem[6]
A number \( X \) is chosen at random from the set \( \{1, 2, 3, \cdot , 12\} \).
A prize is awarded based on the formula \( P = 3X^2 - 5 \).
\begin{subproblems}
    \item Find the expected value of the prize, \( E[P] \).
    \item Find the variance of \( X \).
\end{subproblems}
\( \paren{\text{Hint: } \sum_{i=1}^n i = \frac{n(n+1)}{2} \text{ and } \sum_{i=1}^n i^2 = \frac{n(n+1)(2n+1)}{6}} \)

\vspace{2mm}

\par\noindent\submitsolution
\[
    E[P] = E[3X^2 - 5] = 3E[X^2] - 5
\]

\vspace{2mm}

\par\noindent While \( X \sim Uniform(12) \), the expectation of Uniform Distribution is:
\[
    E[X] = \frac{n+1}{2} = \frac{12+1}{2} = 6.5
\]
and the variance of Uniform Distribution is:
\[
    Var(X) = \frac{n^2 - 1}{12} = \frac{12^2 - 1}{12} = \boxed{\frac{143}{12}} \; \text{b).}
\]
We can calculate \( E[X^2] \) as follows:
\begin{align*}    
    Var(X) &= E[X^2] - \paren{E[x]}^2 \\
    E[X^2] &= Var(X) + \paren{E[x]}^2 \\
    &= \frac{143}{12} + (6.5)^2 \\
    &= \frac{143}{12} + \frac{169}{4} \\
    &= \frac{650}{12} \\
    E[X^2] &= \frac{325}{6}
\end{align*}
, we know that:
\[
    E[P] = 3E[X^2] - 5 = 3 \times \frac{325}{6} - 5 = \frac{325}{2} - 5 = \frac{315}{2} = \boxed{157.5} \; \text{a).}
\]
\end{tosubmit}
% ================================================================================ %


\end{document}