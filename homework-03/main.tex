\documentclass[a4paper, 10pt]{article}
\usepackage{CEDT-Homework-style}

\usepackage{booktabs}
\usepackage{amsmath}
\allowdisplaybreaks

\setlength{\headheight}{14.49998pt}

\begin{document}
\subject[2110205 - Statistics for Computer Engineering]
\hwtitle{3}{Week 3}{6733172621 Patthadon Phengpinij}{ChatGPT (for LaTeX styling and grammar checking)}


% ================================================================================ %
\section{Week 3: Intro to Discrete Random Variable}
% ================================================================================ %



% ================================================================================ %
%                                    Problem 01                                    %
% ================================================================================ %
\begin{problem}
A complex bet on a race has a payout value, \( X \), determined by which tier
of horse wins. The probability mass function (PMF) is given below:
\par\noindent
\begin{table}[h]
    \centering
    \renewcommand{\arraystretch}{1.5}
    \begin{tabular}{|l|c|c|c|c|c|}
        \hline
        \textbf{Payout Value, \( k \) (\$)} & 0 & 10 & 50 & 100 \\ 
        \hline
        \textbf{pmf \( P(X = k) \)} & 0.65 & ? & 0.10 & 0.05 \\
        \hline
    \end{tabular}
    \label{tab:hw03-pb01}
\end{table}

\begin{subproblems}
    \item What is the probability of winning a \$10 payout, \( P(X = 10) \)?
    \item Calculate the cumulative distribution function value at 50, \( F(50) \).
\end{subproblems}
\end{problem}

\begin{solution}
Because, the total PMF value for all possible outcomes is \( 1 \),
\[
    \sum_{\text{all} k} P(X = k) = 1
\]
Thus, we can find the missing PMF value:
\begin{align*}
    1 &= P(X = 0) + P(X = 10) + P(X = 50) + P(X = 100) \\
    1 &= 0.65 + P(X = 10) + 0.10 + 0.05 \\
    P(X = 10) &= 1 - 0.65 - 0.10 - 0.05 \\
    P(X = 10) &= \boxed{0.20} \; \textbf{a).}
\end{align*}
The cumulative distribution function (CDF) value at 50, \( F(50) \),
is the sum of the PMF values for all outcomes less than or equal to 50:
\begin{align*}
    F(50) &= P(X = 0) + P(X = 10) + P(X = 50) \\
    F(50) &= 0.65 + 0.20 + 0.10 \\
    F(50) &= \boxed{0.95} \; \textbf{b).}
\end{align*}
\end{solution}
% ================================================================================ %


% ================================================================================ %
%                                    Problem 02                                    %
% ================================================================================ %
\begin{problem}
Are the following statements true or false?
\begin{subproblems}
    \item The number of a horse's wins in its next 10 races can be modeled by a Geometric distribution.
    \item The PMF of a bet's outcome (a numerical value for win/loss) cannot be negative.
    \item The number of races until a specific long-shot horse wins for the first time is a random variable that can,
    in theory, take on an infinitely large value.
\end{subproblems}
\end{problem}

\begin{solution}
\par\noindent\textbf{a).} \, \( \boxed{\textbf{False.}} \) The number of wins in a fixed number of races (like 10) follows a Binomial distribution, not a Geometric distribution.
\par\noindent\textbf{b).} \, \( \boxed{\textbf{True.}} \) The PMF of a bet's outcome cannot be negative, as it represents probabilities.
\par\noindent\textbf{c).} \, \( \boxed{\textbf{True.}} \) The number of races until the first win can be infinitely large, as it includes the possibility of never winning.
\end{solution}
% ================================================================================ %


% ================================================================================ %
%                                    Problem 03                                    %
% ================================================================================ %
\begin{problem}
For each scenario, what is the most appropriate discrete probability distribution?
(Choose from Bernoulli, Binomial, Geometric, Poisson, or Discrete Uniform).
\begin{subproblems}
    \item The number of wins for the champion horse \textit{Sure Bet} in his next 8 races, given he has a fixed probability of winning each race.
    \item The number of false starts during a full day of 12 races at the track.
    \item The starting gate number (from 1 to 8) assigned to a horse, assuming all gates are assigned randomly.
\end{subproblems}
\end{problem}

\begin{solution}
\vspace{2mm}

\par\noindent\textbf{a).} \( \boxed{\textbf{Binomial Distribution}} \) Each race can be seen as a Bernoulli trial (win or not),
with the same fixed probability of winning, and the outcomes across 8 independent races are counted.

\vspace{2mm}

\par\noindent\textbf{b).} \( \boxed{\textbf{Poisson Distribution}} \) False starts are relatively rare, occur independently,
and can happen multiple times across races. When counting such events over a fixed period (the 12 races),
the Poisson distribution provides the natural model.

\vspace{2mm}

\par\noindent\textbf{c).} \( \boxed{\textbf{Discrete Uniform Distribution}} \) Each gate is equally likely to be assigned,
with no bias toward any specific number, so the assignment follows a discrete uniform distribution over the 8 possible gates.
\end{solution}
% ================================================================================ %


% ================================================================================ %
%                                    Problem 05                                    %
% ================================================================================ %
\begin{tosubmit}
\problem[5]
A long-shot horse named ``Hopeful'' has only a 5\% chance of winning any given race.
Let \( X \) be the number of races he runs until he achieves his first win.
\begin{subproblems}
    \item What is the probability that Hopeful wins on his 4th race?
    \item What is the probability that his first win occurs after the 2nd race?
\end{subproblems}

\par\noindent\submitsolution
The probability of ``Hopeful'' winning the 4th race is the same with the probability of winning--\(\boxed{0.05}\). \textbf{a).}

\noindent The probability that his first win occurs after the 2nd race is
the same with the probability of losing the first 2 races,
\[
    P(\text{first win occurs after 2nd race}) = \paren{0.95 \times 0.95} = \boxed{0.9025} \; \textbf{b).}  \\
\]
\end{tosubmit}
% ================================================================================ %


% ================================================================================ %
%                                    Problem 07                                    %
% ================================================================================ %
\begin{tosubmit}
\problem[7]
A star jockey has a 30\% chance of winning any given race.
Let \( X \) be his number of wins in the 3 morning races, and \( Y \) be his number of wins in the 2 afternoon races.
Assume his performance is independent between races.
Let \( Z = X + Y \) be his total wins for the day.

\begin{subproblems}
    \item What is the distribution of \( Z \) and what are its parameters?
    \item What is the probability that he wins exactly 2 races all day, \( P(Z = 2) \)?
\end{subproblems}

\par\noindent\submitsolution
\textbf{a).} \, The distribution of \( Z \) is a binomial distribution
with parameters \( n = 5 \) (the total number of races) and \( p = 0.3 \) (the probability of winning any given race).

\noindent The probability mass function (PMF) of a binomial distribution is given by
\[
    P(Z = k) = \binom{n}{k} p^k (1-p)^{n-k}
\]
Thus, the probability that he wins exactly 2 races all day is
\[
    P(Z = 2) = \binom{5}{2} (0.3)^2 (0.7)^3 = 10 \cdot 0.09 \cdot 0.343 = \boxed{0.3087} \; \textbf{b).}
\]
\end{tosubmit}
% ================================================================================ %


% ================================================================================ %
%                                    Problem 12                                    %
% ================================================================================ %
\begin{tosubmit}
\problem[12]
A trainer has two horses. Horse A is in a race where its probability of winning is 0.1.
Horse B is in a separate race where its probability of winning is 0.2.
Let \( Z \) be the total number of wins for the trainer.
Find the probability mass function (PMF) for \( Z \).

\vspace{0.5em}
\par\noindent\submitsolution
Let \( X_A\sim \text{Bernoulli}(0.1) \) for Horse A, \( X_B\sim \text{Bernoulli}(0.2) \) for Horse B. \\
Thus, \( Z=X_A+X_B \) with possible values: \( z\in\{0, 1, 2\} \).

\begin{itemize}
    \item \( P(Z=0)=P(A\text{ loses})P(B\text{ loses})=(1-0.1)(1-0.2)=0.9\cdot0.8=0.72 \)
    \item \( P(Z=2)=P(A\text{ wins})P(B\text{ wins})=0.1\cdot0.2=0.02 \)
    \item \( P(Z=1)=1-P(Z=0)-P(Z=2)=1-0.72-0.02=0.26 \)
\end{itemize}

So the PMF is:
\[\boxed{
    P(Z = z) =
    \begin{cases}
    0.72, & z=0\\
    0.26, & z=1\\
    0.02, & z=2\\
    0, & \text{otherwise.}
    \end{cases}
}\]
\end{tosubmit}
% ================================================================================ %


\end{document}