\documentclass[a4paper, 10pt]{article}
\usepackage{../CEDT-Homework-style}

\usepackage{booktabs}
\usepackage{amsmath}
\allowdisplaybreaks

\setlength{\headheight}{14.49998pt}

\begin{document}
\subject[2110205 - Statistics for Computer Engineering]
\hwtitle{3}{}{Week 3}{6733172621 Patthadon Phengpinij}{ChatGPT (for\,\LaTeX\,styling and grammar checking)}


% ================================================================================ %
\section{Week 3: Intro to Discrete Random Variable}
% ================================================================================ %



% ================================================================================ %
%                                    Problem 01                                    %
% ================================================================================ %
\begin{problem}
A complex bet on a race has a payout value, \( X \), determined by which tier
of horse wins. The probability mass function (PMF) is given below:
\par\noindent
\begin{table}[h]
    \centering
    \renewcommand{\arraystretch}{1.5}
    \begin{tabular}{|l|c|c|c|c|c|}
        \hline
        \textbf{Payout Value, \( k \) (\$)} & 0 & 10 & 50 & 100 \\ 
        \hline
        \textbf{pmf \( P(X = k) \)} & 0.65 & ? & 0.10 & 0.05 \\
        \hline
    \end{tabular}
    \label{tab:hw03-pb01}
\end{table}

\begin{subproblems}
    \item What is the probability of winning a \$10 payout, \( P(X = 10) \)?
    \item Calculate the cumulative distribution function value at 50, \( F(50) \).
\end{subproblems}
\end{problem}

\begin{solution}
Because, the total PMF value for all possible outcomes is \( 1 \),
\[
    \sum_{\text{all} k} P(X = k) = 1
\]
Thus, we can find the missing PMF value:
\begin{align*}
    1 &= P(X = 0) + P(X = 10) + P(X = 50) + P(X = 100) \\
    1 &= 0.65 + P(X = 10) + 0.10 + 0.05 \\
    P(X = 10) &= 1 - 0.65 - 0.10 - 0.05 \\
    P(X = 10) &= \boxed{0.20} \; \textbf{a).}
\end{align*}
The cumulative distribution function (CDF) value at 50, \( F(50) \),
is the sum of the PMF values for all outcomes less than or equal to 50:
\begin{align*}
    F(50) &= P(X = 0) + P(X = 10) + P(X = 50) \\
    F(50) &= 0.65 + 0.20 + 0.10 \\
    F(50) &= \boxed{0.95} \; \textbf{b).}
\end{align*}
\end{solution}
% ================================================================================ %

\pagebreak

% ================================================================================ %
%                                    Problem 02                                    %
% ================================================================================ %
\begin{problem}
Are the following statements true or false?
\begin{subproblems}
    \item The number of a horse's wins in its next 10 races can be modeled by a Geometric distribution.
    \item The PMF of a bet's outcome (a numerical value for win/loss) cannot be negative.
    \item The number of races until a specific long-shot horse wins for the first time is a random variable that can,
    in theory, take on an infinitely large value.
\end{subproblems}
\end{problem}

\begin{solution}
\par\noindent\textbf{a).} \, \( \boxed{\textbf{False.}} \) The number of wins in a fixed number of races (like 10) follows a Binomial distribution, not a Geometric distribution.
\par\noindent\textbf{b).} \, \( \boxed{\textbf{True.}} \) The PMF of a bet's outcome cannot be negative, as it represents probabilities.
\par\noindent\textbf{c).} \, \( \boxed{\textbf{True.}} \) The number of races until the first win can be infinitely large, as it includes the possibility of never winning.
\end{solution}
% ================================================================================ %


% ================================================================================ %
%                                    Problem 03                                    %
% ================================================================================ %
\begin{problem}
For each scenario, what is the most appropriate discrete probability distribution?
(Choose from Bernoulli, Binomial, Geometric, Poisson, or Discrete Uniform).
\begin{subproblems}
    \item The number of wins for the champion horse \textit{Sure Bet} in his next 8 races, given he has a fixed probability of winning each race.
    \item The number of false starts during a full day of 12 races at the track.
    \item The starting gate number (from 1 to 8) assigned to a horse, assuming all gates are assigned randomly.
\end{subproblems}
\end{problem}

\begin{solution}
\vspace{2mm}

\par\noindent\textbf{a).} \( \boxed{\textbf{Binomial Distribution}} \) Each race can be seen as a Bernoulli trial (win or not),
with the same fixed probability of winning, and the outcomes across 8 independent races are counted.

\vspace{2mm}

\par\noindent\textbf{b).} \( \boxed{\textbf{Poisson Distribution}} \) False starts are relatively rare, occur independently,
and can happen multiple times across races. When counting such events over a fixed period (the 12 races),
the Poisson distribution provides the natural model.

\vspace{2mm}

\par\noindent\textbf{c).} \( \boxed{\textbf{Discrete Uniform Distribution}} \) Each gate is equally likely to be assigned,
with no bias toward any specific number, so the assignment follows a discrete uniform distribution over the 8 possible gates.
\end{solution}
% ================================================================================ %

\pagebreak

% ================================================================================ %
%                                    Problem 04                                    %
% ================================================================================ %
\begin{problem}
The horse ``Gallant Prince'' wins 20\% of the races it enters.
Let \( X \) be the number of wins for Gallant Prince in his next 4 races.
Assume the outcomes are independent.
\begin{subproblems}
    \item Calculate the probability that he wins exactly one race, \( P(X = 1) \).
    \item Calculate the probability that he wins no races, \( P(X = 0) \).
\end{subproblems}
\end{problem}

\begin{solution}
The probability that Gallant Prince wins exactly \( k \) races out of \( n \) races
can be calculated using the Binomial distribution formula:
\[
    P(X = k) = \binom{n}{k} p^k (1-p)^{n-k}
\]
where \( n = 4 \) (the number of races), and \( p = 0.2 \) (the probability of winning a race).

\par\noindent\textbf{a).} \, The probability that Gallant Prince wins exactly \textbf{one race} out of 4 is given by
\begin{align*}
    P(X = 1) &= \binom{4}{1} (0.2)^1 (0.8)^{4-1} \\
    &= 4 \cdot 0.2 \cdot (0.8)^3 \\
    &= 4 \cdot 0.2 \cdot 0.512 \\
    &= 0.4096 \\
    &= \boxed{0.4096} \; \textbf{a).}
\end{align*}

\par\noindent\textbf{b).} \, The probability that Gallant Prince wins \textbf{no races} out of 4 is given by
\begin{align*}
    P(X = 0) &= \binom{4}{0} (0.2)^0 (0.8)^{4-0} \\
    &= 1 \cdot 1 \cdot (0.8)^4 \\
    &= (0.8)^4 \\
    &= 0.4096 \\
    &= \boxed{0.4096} \; \textbf{b).}
\end{align*}

\end{solution}
% ================================================================================ %


% ================================================================================ %
%                                    Problem 05                                    %
% ================================================================================ %
\begin{tosubmit}
\problem[5]
A long-shot horse named ``Hopeful'' has only a 5\% chance of winning any given race.
Let \( X \) be the number of races he runs until he achieves his first win.
\begin{subproblems}
    \item What is the probability that Hopeful wins on his 4th race?
    \item What is the probability that his first win occurs after the 2nd race?
\end{subproblems}

\par\noindent\submitsolution
The probability of ``Hopeful'' winning the 4th race is the same with the probability of winning--\(\boxed{0.05}\). \textbf{a).}

\noindent The probability that his first win occurs after the 2nd race is
the same with the probability of losing the first 2 races,
\[
    P(\text{first win occurs after 2nd race}) = \paren{0.95 \times 0.95} = \boxed{0.9025} \; \textbf{b).}  \\
\]
\end{tosubmit}
% ================================================================================ %

\pagebreak

% ================================================================================ %
%                                    Problem 06                                    %
% ================================================================================ %
\begin{problem}
The number of jockey suspensions for misconduct at a large racetrack
follows a Poisson distribution with an average of 2 suspensions per week.
\begin{subproblems}
    \item What is the probability of having exactly one suspension in a given week?
    \item What is the probability of having no suspensions in a given week?
\end{subproblems}
\end{problem}

\begin{solution}
The number of jockey suspensions for misconduct at a large racetrack follows a Poisson distribution
with parameter \( \lambda = 2 \) (the average number of suspensions per week).

The probability of having exactly \( k \) events (suspensions) in a Poisson distribution is given by the formula:
\[
    P(X = k) = \frac{\lambda^k e^{-\lambda}}{k!}
\]

\par\noindent\textbf{a).} \, For exactly one suspension (\( k = 1 \)):
\begin{align*}
    P(X = 1) &= \frac{2^1 e^{-2}}{1!} \\
    &= 2 e^{-2} \\
    &\approx 2 \cdot 0.1353 \\
    &\approx \boxed{0.2706} \; \textbf{a).}
\end{align*}

\par\noindent\textbf{b).} \, For no suspensions (\( k = 0 \)):
\begin{align*}
    P(X = 0) &= \frac{2^0 e^{-2}}{0!} \\
    &= e^{-2} \\
    &\approx \boxed{0.1353} \; \textbf{b).}
\end{align*}
\end{solution}
% ================================================================================ %


% ================================================================================ %
%                                    Problem 07                                    %
% ================================================================================ %
\begin{tosubmit}
\problem[7]
A star jockey has a 30\% chance of winning any given race.
Let \( X \) be his number of wins in the 3 morning races, and \( Y \) be his number of wins in the 2 afternoon races.
Assume his performance is independent between races.
Let \( Z = X + Y \) be his total wins for the day.

\begin{subproblems}
    \item What is the distribution of \( Z \) and what are its parameters?
    \item What is the probability that he wins exactly 2 races all day, \( P(Z = 2) \)?
\end{subproblems}

\par\noindent\submitsolution
The distribution of \( Z \) is a binomial distribution with parameters \( n = 5 \) (the total number of races)
and \( p = 0.3 \) (the probability of winning any given race).
\par\noindent Thus, \( \boxed{Z \sim \text{Binomial}(5,\, 0.3)} \). \textbf{a).}

\vspace{2mm}

\par\noindent The probability mass function (PMF) of a binomial distribution is given by
\[
    P(Z = k) = \binom{n}{k} p^k (1-p)^{n-k}
\]
Thus, the probability that he wins exactly 2 races all day is
\[
    P(Z = 2) = \binom{5}{2} (0.3)^2 (0.7)^3 = 10 \cdot 0.09 \cdot 0.343 = \boxed{0.3087} \; \textbf{b).}
\]
\end{tosubmit}
% ================================================================================ %

\pagebreak

% ================================================================================ %
%                                    Problem 08                                    %
% ================================================================================ %
\begin{problem}
In an 8-horse race, a gambler places a simple \underline{win} bet on a single horse, ``Lucky Number 7''.
Assume all horses have an equal chance of winning.
Let the random variable \( X = 1 \) if Lucky Number 7 wins and \( X = 0 \) if it loses.
\begin{subproblems}
    \item What is the name of the probability distribution for \( X \)?
    \item Find the probability mass function (PMF) of \( X \).
\end{subproblems}
\end{problem}

\begin{solution}
\par\noindent\textbf{a).} \, The probability distribution for \( X \) is a \( \boxed{\textbf{Bernoulli distribution}} \).
Because \( X \) can take on only two possible outcomes: winning (1) or losing (0).

\vspace{2mm}

\par\noindent\textbf{b).} \, The probability mass function (PMF) of \( X \) is given by:
\[\boxed{
    P(X = x) =
    \begin{cases}
    \frac{1}{8}, & x=1\\
    \frac{7}{8}, & x=0\\
    0, & \text{otherwise.}
    \end{cases}
}\]
\end{solution}
% ================================================================================ %


% ================================================================================ %
%                                    Problem 09                                    %
% ================================================================================ %
\begin{problem}
A `Daily Double' bet requires picking the winners of two consecutive races.
A gambler chooses one horse in the first race (8 horses, all equal chance) and
one horse in the second race (10 horses, all equal chance).
Let \( Z = 1 \) if the gambler wins the bet and \( Z = 0 \) if they lose. What is \( P(Z = 1) \)?
\end{problem}

\begin{solution}
To win the `Daily Double' bet, the gambler must correctly pick the winners of both:
\par 1. \textbf{The first race}: There are 8 horses, so the probability of picking the winning horse is \( \frac{1}{8} \).
\par 2. \textbf{The second race}: There are 10 horses, so the probability of picking the winning horse is \( \frac{1}{10} \).

\vspace{2mm}

Thus, the probability of winning the `Daily Double' bet is:
\[
    P(Z = 1) = P(\text{win first}) \times P(\text{win second}) = \frac{1}{8} \times \frac{1}{10} = \frac{1}{80} = \boxed{0.0125}.
\]

So that, \( P(Z = 0) = 1 - P(Z = 1) = 1 - 0.0125 = 0.9875 \).
\[
    P(Z = z) =
    \begin{cases}
    0.0125, & z=1\\
    0.9875, & z=0\\
    0, & \text{otherwise.}
    \end{cases}
\]

\end{solution}
% ================================================================================ %

\pagebreak

% ================================================================================ %
%                                    Problem 10                                    %
% ================================================================================ %
\begin{problem}
The number of photo-finishes on a clear day is a Poisson random variable with mean 1.5.
The number on a rainy day is an independent Poisson random variable with mean 3.
Let \( W \) be the total number of photo-finishes from one clear day and one rainy day.
\begin{subproblems}
    \item What is the distribution of \( W \) and its mean?
    \item What is the probability of observing a total of exactly 4 photo-finishes?
\end{subproblems}
\end{problem}

\begin{solution}
Let \( X \) be the number of photo-finishes on a clear day, and \( Y \) be the number on a rainy day.
Since \( X \) and \( Y \) are independent Poisson random variables, their sum \( W = X + Y \) is also a Poisson random variable.
The mean of \( W \) is the sum of the means of \( X \) and \( Y \):
\[
    \lambda_W = \lambda_X + \lambda_Y = 1.5 + 3 = 4.5.
\]
Thus, \( \boxed{W \sim \text{Poisson}(4.5)} \). \textbf{a).}

\vspace{2mm}

\par\noindent The probability of observing exactly \( k \) events in a Poisson distribution is given by:
\[
    P(W = k) = \frac{\lambda_W^k e^{-\lambda_W}}{k!}
\]
For exactly 4 photo-finishes (\( k = 4 \)):
\begin{align*}
    P(W = 4) &= \frac{\paren{4.5}^4 e^{-4.5}}{4!} \\
    &= \frac{410.0625 e^{-4.5}}{24} \\
    &\approx \frac{410.0625 \times 0.0111}{24} \\
    &\approx \frac{4.5517}{24} \\
    &\approx \boxed{0.1897} \; \textbf{b).}
\end{align*}
\end{solution}
% ================================================================================ %


% ================================================================================ %
%                                    Problem 11                                    %
% ================================================================================ %
\begin{problem}
The CDF for the number of wins (\( X \)) for a young horse in its first season is given below:
\[
    F(x) =
    \begin{cases}
    0, & x < 0\\
    0.5, & 0 \leq x < 1\\
    0.8, & 1 \leq x < 2\\
    0.95, & 2 \leq x < 3\\
    1, & x \geq 3
    \end{cases}
\]
\begin{subproblems}
    \item Find the probability that the horse wins exactly 2 races, \( P(X = 2) \).
    \item Find the probability that the horse wins more than 1 race, \( P(X > 1) \).
\end{subproblems}
\end{problem}

\begin{solution}
\par\noindent\textbf{a).} \, The probability that the horse wins exactly 2 races, \( P(X = 2) \), can be found using the CDF:
\[
    P(X = 2) = F(2) - F(1) = 0.95 - 0.8 = \boxed{0.15} \; \textbf{a).}
\]

\vspace{2mm}

\par\noindent\textbf{b).} \, The probability that the horse wins more than 1 race, \( P(X > 1) \), is given by:
\[
    P(X > 1) = 1 - P(X \leq 1) = 1 - F(1) = 1 - 0.8 = \boxed{0.2} \; \textbf{b).}
\]
\end{solution}
% ================================================================================ %


% ================================================================================ %
%                                    Problem 12                                    %
% ================================================================================ %
\begin{tosubmit}
\problem[12]
A trainer has two horses. Horse A is in a race where its probability of winning is 0.1.
Horse B is in a separate race where its probability of winning is 0.2.
Let \( Z \) be the total number of wins for the trainer.
Find the probability mass function (PMF) for \( Z \).

\vspace{0.5em}
\par\noindent\submitsolution
Let \( X_A\sim \text{Bernoulli}(0.1) \) for Horse A, \( X_B\sim \text{Bernoulli}(0.2) \) for Horse B. \\
Thus, \( Z=X_A+X_B \) with possible values: \( z\in\{0, 1, 2\} \).

\begin{itemize}
    \item \( P(Z=0)=P(A\text{ loses})P(B\text{ loses})=(1-0.1)(1-0.2)=0.9\cdot0.8=0.72 \)
    \item \( P(Z=2)=P(A\text{ wins})P(B\text{ wins})=0.1\cdot0.2=0.02 \)
    \item \( P(Z=1)=1-P(Z=0)-P(Z=2)=1-0.72-0.02=0.26 \)
\end{itemize}

So the PMF is:
\[\boxed{
    P(Z = z) =
    \begin{cases}
    0.72, & z=0\\
    0.26, & z=1\\
    0.02, & z=2\\
    0, & \text{otherwise.}
    \end{cases}
}\]
\end{tosubmit}
% ================================================================================ %


% ================================================================================ %
%                                    Problem 13                                    %
% ================================================================================ %
\begin{problem}
Calculate the following probabilities for a racing season:
\begin{subproblems}
    \item A jockey has a 10\% win rate. Find the probability he wins at least once in his next 5 races.
    \( P(X \geq 1) \) for \( X \sim Binomial(5, 0.1) \).
    \item The average number of scratches per day is 2.5. Find the probability there are fewer than 2 scratches on a given day.
    \( P(Y < 2) \) for \( Y \sim Poisson(2.5) \).
\end{subproblems}
\end{problem}

\begin{solution}
\par\noindent\textbf{a).} \, \( X \sim Binomial(5, 0.1) \)
\begin{align*}
    P(X \geq 1) &= 1 - P(X = 0) \\
    &= 1 - \paren{\binom{5}{0} (0.1)^0 (0.9)^5} \\
    &= 1 - \paren{1 \cdot 1 \cdot (0.9)^5} \\
    &= 1 - (0.59049) \\
    &= \boxed{0.40951} \; \textbf{a).}
\end{align*}

\vspace{2mm}

\par\noindent\textbf{b).} \, \( Y \sim Poisson(2.5) \)
\begin{align*}
    P(Y < 2) &= P(Y = 0) + P(Y = 1) \\
    &= \frac{(2.5)^0 e^{-2.5}}{0!} + \frac{(2.5)^1 e^{-2.5}}{1!} \\
    &= e^{-2.5} + 2.5 e^{-2.5} \\
    &= (1 + 2.5) e^{-2.5} \\
    &= 3.5 e^{-2.5} \\
    &\approx 3.5 \cdot 0.0821 \\
    &\approx \boxed{0.28735} \; \textbf{b).}
\end{align*}
\end{solution}
% ================================================================================ %


% ================================================================================ %
%                                    Problem 14                                    %
% ================================================================================ %
\begin{problem}
A jockey is scheduled for 7 races in a festival. His probability of winning any single race is 0.15.
What is the probability he wins at least two races during the festival?
\end{problem}

\begin{solution}
The probability that the jockey wins \( n \) races out of \( k \) races
can be calculated using the Binomial distribution formula:
\[
    P(X = n) = \binom{k}{n} p^n (1-p)^{k-n}
\]
In this case, \( k = 7 \) (the number scheduled races), and \( p = 0.15 \) (the probability of winning).

\par\noindent The probability that he wins at least 2 races is:
\begin{align*}
    P(X \geq 2) &= 1 - P(X < 2) \\
    &= 1 - (P(X = 0) + P(X = 1)) \\
    &= 1 - \paren{ \binom{7}{0} (0.15)^0 (0.85)^7 + \binom{7}{1} (0.15)^1 (0.85)^6 } \\
    &= 1 - \paren{ 1 \cdot 1 \cdot (0.85)^7 + 7 \cdot 0.15 \cdot (0.85)^6 } \\
    &= 1 - \paren{ (0.85)^7 + 1.05 \cdot (0.85)^6 } \\
    &\approx 1 - \paren{ 0.3206 + 1.05 \cdot 0.3771 } \\
    &\approx 1 - (0.3206 + 0.3960) \\
    &\approx 1 - 0.7166 \\
    &\approx \boxed{0.2834}
\end{align*}
\end{solution}
% ================================================================================ %


% ================================================================================ %
%                                    Problem 15                                    %
% ================================================================================ %
\begin{problem}
In a 4-horse race, the probabilities of winning for each horse are:
Horse 1: 0.4, Horse 2: 0.3, Horse 3: 0.2, Horse 4: 0.1.
Let the random variable \( X \) be the number of the winning horse.
\begin{subproblems}
    \item Write out the probability mass function (PMF) for \( X \).
    \item Find the probability that the winning horse's number is odd.
\end{subproblems}
\end{problem}

\begin{solution}
\par\noindent\textbf{a).} \, The probability mass function (PMF) for \( X \) is:
\[\boxed{
    P(X = x) =
    \begin{cases}
    0.4, & x=1\\
    0.3, & x=2\\
    0.2, & x=3\\
    0.1, & x=4\\
    0, & \text{otherwise.}
    \end{cases}
}\]

\vspace{2mm}

\par\noindent\textbf{b).} \, The probability that the winning horse's number is odd (Horse 1 or Horse 3) is:
\[
    P(X \text{ is odd}) = P(X = 1) + P(X = 3) = 0.4 + 0.2 = \boxed{0.6}
\]
\end{solution}
% ================================================================================ %


\end{document}