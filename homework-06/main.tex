\documentclass[a4paper, 10pt]{article}
\usepackage{../CEDT-Homework-style}

\usepackage{amsmath}
\allowdisplaybreaks

\setlength{\headheight}{14.49998pt}

\begin{document}
\subject[2110205 - Statistics for Computer Engineering]
\hwtitle{6}{}{Week 7}{6733172621 Patthadon Phengpinij}{ChatGPT (for\,\LaTeX\,styling and grammar checking)}


% ================================================================================ %
\section{Law of Large Number (LLN)}
% ================================================================================ %



% ================================================================================ %
%                                    Problem 01                                    %
% ================================================================================ %
\begin{problem}
Let \( X \) be a Gaussian distribution with mean 5 and variance 4.
\( Y \) be a uniform distribution with mean 5 and variance 4.
\( Z = X + Y \). What should be the expected value and variance of \( Z \)?
\end{problem}

\begin{solution}
Coming Soon...
\end{solution}
% ================================================================================ %



% ================================================================================ %
\section{Central Limit Theorem (CLT) and Joint Distribution}
% ================================================================================ %



% ================================================================================ %
%                                    Problem 08                                    %
% ================================================================================ %
\begin{problem}[8]
We have two coins \( A \) and \( B \).
\( A \) has a 0.7 probability to land heads.
\( B \) has a 0.5 probability to land heads.
We use the two coins to play a game of coin tossing.
Both coins are tossed at the same time, if they both land heads, we win.
Let \( W \) be the number of wins after we play the game 100,000 times.
What is the distribution of \( W \)? Answer both the distribution name and its parameters.
Compute \( P(W > 36, 000) \).
\end{problem}

\begin{solution}
Coming Soon...
\end{solution}
% ================================================================================ %

\newpage

% ================================================================================ %
%                                    Problem 16                                    %
% ================================================================================ %
\begin{tosubmit}
\problem[16]
\[
    P(X, Y) = \begin{cases}
        c \cdot (x+y) & \text{ if } y \geq x, 0 \leq x \leq 1, 0 \leq y \leq 1 \\
        0 & \text{otherwise}
    \end{cases}
\]
Find the constant \( c \), \( P(X) \), \( P(Y) \), \( E[X] \), and \( E[Y] \).

\par\noindent\submitsolution
Given the joint probability density function (pdf) \( P(X, Y) = f_{X,Y}(x, y) \), the integral over the entire support must equal 1:
\begin{align*}
    1 &= \int_0^1 \int_x^1 f_{X,Y}(x, y) \, dy \, dx \\
    &= \int_0^1 \int_x^1 c(x + y) \, dy \, dx \\
    &= \int_0^1 c \sqbracket{xy + \frac{y^2}{2}}_{y=x}^{y=1} \, dx \\
    &= \int_0^1 c \sqbracket{x + \frac{1}{2} - x^2 - \frac{x^2}{2}} \, dx \\
    &= c \int_0^1 \sqbracket{\frac{1}{2} + x - \frac{3}{2}x^2} \, dx \\
    &= c \sqbracket{\frac{1}{2}x + \frac{1}{2}x^2 - \frac{1}{2}x^3}_{0}^{1} \\
    1 &= c \cdot \frac{1}{2} \\
    \implies c &= 2
\end{align*}

To find the marginal pdfs \( f_X(x) \) and \( f_Y(y) \): \newline
For \( 0 \leq x \leq 1 \):
\begin{align*}
    f_X(x) &= \int_x^1 f_{X,Y}(x, y) \, dy \\
    &= \int_x^1 2(x + y) \, dy \\
    &= 2 \sqbracket{xy + \frac{y^2}{2}}_{y=x}^{y=1} \\
    &= 2 \sqbracket{x + \frac{1}{2} - x^2 - \frac{x^2}{2}} \\
    f_X(x) &= 1 + 2x - 3x^2
\end{align*}

For \( 0 \leq y \leq 1 \):
\begin{align*}
    f_Y(y) &= \int_0^y f_{X,Y}(x, y) \, dx \\
    &= \int_0^y 2(x + y) \, dx \\
    &= 2 \sqbracket{\frac{x^2}{2} + yx}_{0}^{y} \\
    &= 2 \sqbracket{\frac{y^2}{2} + y^2} \\
    f_Y(y) &= 3y^2
\end{align*}

\pagebreak

To find the expected values \( E[X] \) and \( E[Y] \):
\begin{align*}
    E[X] &= \int_0^1 x f_X(x) \, dx \\
    &= \int_0^1 x (1 + 2x - 3x^2) \, dx \\
    &= \int_0^1 (x + 2x^2 - 3x^3) \, dx \\
    &= \sqbracket{\frac{x^2}{2} + \frac{2x^3}{3} - \frac{3x^4}{4}}_{0}^{1} \\
    E[X] &= \frac{1}{2} + \frac{2}{3} - \frac{3}{4} = \frac{5}{12}
\end{align*}

And,
\begin{align*}
    E[Y] &= \int_0^1 y f_Y(y) \, dy \\
    &= \int_0^1 y (3y^2) \, dy \\
    &= \int_0^1 3y^3 \, dy \\
    &= \sqbracket{\frac{3y^4}{4}}_{0}^{1} \\
    E[Y] &= \frac{3}{4}
\end{align*}

Thus, the results are:
\[ \boxed{c = 2}, \]

\[ \boxed{f_X(x) = \begin{cases}
    1 + 2x - 3x^2 & \text{ for } 0 \leq x \leq 1 \\
    0 & \text{ otherwise }
\end{cases}}, \]

\[ \boxed{f_Y(y) = \begin{cases}
    3y^2 & \text{ for } 0 \leq y \leq 1 \\
    0 & \text{ otherwise }
\end{cases}}, \]

\[ \boxed{E[X] = \frac{5}{12}}, \quad \boxed{E[Y] = \frac{3}{4}} \]
\end{tosubmit}
% ================================================================================ %


\end{document}